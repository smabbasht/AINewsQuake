
\documentclass[aspectratio=169]{beamer}

% ---------- Theme ----------
\usetheme{Madrid}
\usecolortheme{default}
\usefonttheme{professionalfonts}

% ---------- Packages ----------
\usepackage{graphicx}
\usepackage{booktabs}
\usepackage{xcolor}
\usepackage{hyperref}
\usepackage{amsmath}
\usepackage{ragged2e}

% ---------- Colors ----------
\definecolor{Primary}{HTML}{1F4E79}
\definecolor{Secondary}{HTML}{2E75B6}

\setbeamercolor{title}{fg=white,bg=Primary}
\setbeamercolor{frametitle}{fg=white,bg=Primary}
\setbeamercolor{structure}{fg=Secondary}

\hypersetup{
  colorlinks=true,
  linkcolor=Primary,
  urlcolor=Secondary
}


% ---------- Custom Footline (prevents truncation) ----------
\setbeamertemplate{footline}{
  \leavevmode%
  \hbox{%
  \begin{beamercolorbox}[wd=.80\paperwidth,ht=2.4ex,dp=1.2ex,leftskip=0.8em]{author in head/foot}%
    \usebeamerfont{author in head/foot}\insertshortauthor\hspace{0.6em}\textcolor{Secondary}{\footnotesize\textbar}\hspace{0.6em}\insertshortinstitute
  \end{beamercolorbox}%
  \begin{beamercolorbox}[wd=.20\paperwidth,ht=2.4ex,dp=1.2ex,right,rightskip=0.8em]{title in head/foot}%
    \usebeamerfont{title in head/foot}\insertshorttitle\hspace{0.8em}\insertframenumber/\inserttotalframenumber
  \end{beamercolorbox}}%
  \vskip0pt%
}
\setbeamertemplate{navigation symbols}{}

% ---------- Commands ----------
\newcommand{\projectname}{\textbf{AINewsQuake}}
\newcommand{\repo}{\href{https://github.com/smabbasht/ainewsquake}{github.com/smabbasht/ainewsquake}}

% ---------- Logo ----------
\titlegraphic{
\vspace{-0.2cm}
\includegraphics[height=1.3cm]{Milano-Bicocca_University_logo_on_transparent_background.svg.png}
}

% ---------- Title Info ----------
\title[\projectname]{\projectname\\\small Price Chart with News Annotations}
\subtitle{Data Management Project Presentation}
\author[Syed Muhammad Abbas Haider Taqvi \textbar\ Umeir Mohamed \textbar\ Mohammad Amin Saberi]{Syed Muhammad Abbas Haider Taqvi \and Umeir Mohamed \and Mohammad Amin Saberi}
\institute[University of Milano-Bicocca (UNIMIB)]{University of Milano-Bicocca (UNIMIB)}
\date{\today}

\begin{document}

% ---------- Title Slide ----------
\begin{frame}
  \titlepage
\end{frame}

% ---------- Agenda ----------
\begin{frame}{Agenda}
\begin{itemize}
  \item Motivation \& Objectives
  \item Research Questions
  \item Data Sources \& Acquisition
  \item Storage \& Database Design
  \item Data Quality \& Integration
  \item Dashboard \& Analysis
  \item Limitations \& Future Work
  \item Conclusion
\end{itemize}
\end{frame}

% ---------- Motivation ----------
\begin{frame}{Motivation}
\justifying
Financial markets react quickly to information. AI-related headlines can trigger significant short-term price movements, volatility spikes, and abnormal trading volumes.

\vspace{0.4cm}
\textbf{Goal:} Build a reproducible data management pipeline that integrates \textbf{AI news sentiment} with \textbf{high-frequency market data} and enables interactive exploration of their relationship.
\end{frame}

% ---------- Objectives ----------
\begin{frame}{Objectives}
\begin{itemize}
  \item Acquire heterogeneous data: \textbf{news events} + \textbf{1-min OHLCV}.
  \item Enrich news with \textbf{sentiment scores} (VADER compound).
  \item Store large time-series data efficiently using \textbf{TimescaleDB}.
  \item Provide reproducible ETL with \textbf{idempotent loads} and validation.
  \item Enable analysis via \textbf{SQL queries} and a \textbf{Streamlit dashboard}.
\end{itemize}
\end{frame}

% ---------- Research Questions ----------
\begin{frame}{Research Questions}
\begin{block}{RQ1}
How does AI-related news sentiment correlate with intraday volatility and price movement?
\end{block}
\begin{block}{RQ2}
Do certain AI-centric stocks react more strongly to positive/negative headlines?
\end{block}
\begin{block}{RQ3}
How complete and reliable is the integrated dataset (quality + integration coverage)?
\end{block}
\end{frame}

% ---------- Data Sources ----------
\begin{frame}{Data Sources}
\begin{columns}[T,onlytextwidth]
\column{0.5\textwidth}
\textbf{Finnhub (News API)}
\begin{itemize}
  \item Historical company news (2025)
  \item Headline, timestamp, source
  \item Rate limits + pagination (250 items/request)
\end{itemize}

\column{0.5\textwidth}
\textbf{Databento (Market Data)}
\begin{itemize}
  \item 1-minute OHLCV bars (2025)
  \item High reliability for intraday bars
  \item Large data volume $\rightarrow$ time-series DB needed
\end{itemize}
\end{columns}
\end{frame}

% ---------- Acquisition Constraints ----------
\begin{frame}{Acquisition Strategy}
\begin{itemize}
  \item \textbf{Backfill strategy:} fetch news backwards due to Finnhub limits.
  \item \textbf{Validation:} Pydantic schemas ensure type correctness.
  \item \textbf{Repeatable runs:} ETL is designed to be idempotent.
\end{itemize}
\end{frame}

% ---------- Storage ----------
\begin{frame}{Storage: TimescaleDB}
\begin{itemize}
  \item PostgreSQL + time-series extension.
  \item \textbf{Hypertables} partition tick data by time.
  \item Efficient range queries + indexing on \texttt{(time, ticker)}.
  \item \textbf{Compression policy} for older data (e.g., > 7 days).
\end{itemize}

\vspace{0.3cm}
\textbf{Why this matters:} 1-min OHLCV for 10 tickers across a year $\rightarrow$ millions of rows.
\end{frame}

% ---------- Schema ----------
\begin{frame}{Database Schema (Logical)}
\begin{itemize}
  \item \textbf{\texttt{ai\_news\_events}} (relational table)
  \begin{itemize}
    \item \texttt{event\_id (PK)}, \texttt{ticker}, \texttt{published\_at}, \texttt{headline}, \texttt{source}
    \item \texttt{sentiment\_score} $\in [-1,1]$
  \end{itemize}
  \vspace{0.2cm}
  \item \textbf{\texttt{market\_ticks}} (TimescaleDB hypertable)
  \begin{itemize}
    \item primary key: \texttt{(time, ticker)}
    \item \texttt{open, high, low, close, volume}
  \end{itemize}
\end{itemize}
\end{frame}

% ---------- Data Quality ----------
\begin{frame}{Data Quality}
\begin{itemize}
  \item \textbf{Completeness:} missing fields (timestamps, OHLCV, headline).
  \item \textbf{Validity:} sentiment score range, non-negative volume.
  \item \textbf{Uniqueness:} unique event IDs; unique \texttt{(time, ticker)}.
  \item \textbf{Consistency:} timezone normalization (UTC) across sources.
\end{itemize}
\end{frame}

% ---------- Idempotency ----------
\begin{frame}{Idempotent ETL \& De-duplication}
\begin{itemize}
  \item News insert: \texttt{ON CONFLICT DO NOTHING}
  \item Market insert: \texttt{ON CONFLICT DO UPDATE}
  \item Ensures reproducibility and safe re-runs.
  \item Prevents duplicate rows and supports partial backfills.
\end{itemize}
\end{frame}

% ---------- Temporal Data Integration (Professor's Feedback) ----------
\begin{frame}{Temporal Data Integration Strategy}
\begin{columns}[T,onlytextwidth]
\column{0.5\textwidth}
\textbf{The Challenge:}
\begin{itemize}
  \item News occurs 24/7 (continuous).
  \item Markets trade Mon-Fri, 9:30-16:00 (discrete).
  \item Naive joins lose off-hours news (weekends, overnight).
\end{itemize}

\column{0.5\textwidth}
\textbf{Our Solution:}
\begin{itemize}
  \item **Forward-Fill Alignment:** Map events to the \textit{next available} trading tick.
  \item \textit{Example:} Saturday News $\rightarrow$ Impact measured from Monday 9:30 AM Open.
  \item **Dynamic Baselines:** Volume baseline calculated on \textit{trading ticks} (last 120 minutes), not wall-clock time.
\end{itemize}
\end{columns}

\vspace{0.4cm}
\textbf{Outcome:} Preserved 100\% of off-market news events while maintaining statistical validity of impact metrics.
\end{frame}

% ---------- Integration Quality Metrics ----------
\begin{frame}{Integration Quality Metrics}
\begin{itemize}
  \item \textbf{Coverage:} 100\% of News Events mapped to a reaction window.
  \item \textbf{Latency:} Time delta between publication and market reaction start (0 min for intraday, variable for off-hours).
  \item \textbf{Volume Normalization:} Ratio of post-news volume vs. pre-news trading average (handling overnight gaps).
\end{itemize}
\end{frame}

% ---------- Dashboard ----------
\begin{frame}{Streamlit Dashboard}
\begin{itemize}
  \item Interactive candlestick chart with sentiment-coded news markers.
  \item Hover tooltips show headline, sentiment score, and time.
  \item Filters: ticker selection + date range.
  \item Summary metrics: number of events, average sentiment, volume, price change.
\end{itemize}

\vspace{0.3cm}
\textbf{Demo:} \textit{Price Chart with News Annotations}
\end{frame}

% ---------- Analysis Examples ----------
\begin{frame}{Analysis Examples}
\begin{itemize}
  \item Identify volatility clusters around major AI headlines.
  \item Compare average sentiment vs returns across tickers.
  \item Detect extreme price moves and inspect associated news.
\end{itemize}
\end{frame}

% ---------- Reproducibility ----------
\begin{frame}{Reproducibility}
\begin{itemize}
  \item Docker Compose for TimescaleDB deployment.
  \item \texttt{uv} for dependency management.
  \item \texttt{.env} configuration for API keys.
  \item One-command ETL runs + Streamlit launch.
\end{itemize}

\vspace{0.2cm}
\textbf{Repository:} \repo
\end{frame}

% ---------- Limitations ----------
\begin{frame}{Limitations}
\begin{itemize}
  \item Free-tier API limits can affect backfill completeness.
  \item VADER sentiment is generic; domain-specific models may improve quality.
  \item Correlation $\neq$ causation; causal inference is future work.
\end{itemize}
\end{frame}

% ---------- Future Work ----------
\begin{frame}{Future Work}
\begin{itemize}
  \item Add anomaly detection to identify ``quake'' events systematically.
  \item Build automated DQ dashboards (metrics over time).
  \item Extend to more assets (crypto/forex) and additional APIs.
  \item Improve temporal integration with richer pre/post windows.
\end{itemize}
\end{frame}

% ---------- Conclusion ----------
\begin{frame}{Conclusion}
\begin{itemize}
  \item End-to-end pipeline: acquisition $\rightarrow$ storage $\rightarrow$ profiling $\rightarrow$ integration $\rightarrow$ analysis.
  \item Efficient time-series database design using TimescaleDB.
  \item Robust ETL with validation + idempotent loads.
  \item Interactive dashboard for exploring sentiment-volatility relationships.
\end{itemize}

\vspace{0.3cm}
\centering
\textbf{Thank you!}\\
\small Questions?
\end{frame}

\end{document}
